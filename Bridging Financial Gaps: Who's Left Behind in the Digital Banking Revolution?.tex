% This LaTeX document needs to be compiled with XeLaTeX.
\documentclass[10pt]{article}
\usepackage[utf8]{inputenc}
\usepackage{ucharclasses}
\usepackage{amsmath}
\usepackage{amsfonts}
\usepackage{amssymb}
\usepackage[version=4]{mhchem}
\usepackage{stmaryrd}
\usepackage{hyperref}
\hypersetup{colorlinks=true, linkcolor=blue, filecolor=magenta, urlcolor=cyan,}
\urlstyle{same}
\usepackage{polyglossia}
\usepackage{fontspec}
\setmainlanguage{english}
\setotherlanguages{hindi}
\newfontfamily\hindifont{Noto Serif Devanagari}
\newfontfamily\lgcfont{CMU Serif}
\setDefaultTransitions{\lgcfont}{}
\setTransitionsFor{Hindi}{\hindifont}{\lgcfont}

\title{Bridging Financial Gaps: Who's Left Behind in the Digital Banking Revolution? }

\author{}
\date{}


\begin{document}
\maketitle
\section*{Why must it be inclusive?}
Firstly, what does it mean for the financial services having "inclusivity"? Although there is a lack of standard definition, several international institutions share the same concept about the fundamental criteria, that is making sure everyone can use basic money services. This includes people who are at the bottom of the pyramid (low and irregular income, isolated, living with disabilities, undocumented workers, disadvantaged communities), meaning that there will be problems faced by those without access to financial services, such as income inequality, lack of stable housing and even systemic poverty.

However, things are improving...

OJK (Otoritas Jasa Keuangan) surveys prove that financial inclusion has made significant progress, especially in Indonesia. Financial inclusion reached $88.7 \%$ in 2023 , a notable increase from $85.1 \%$ the previous year (2022). This achievement can be attributed to the SNKI (Sekretariat Dewan Nasional Keuangan Inklusif), initiated in 2016, which has successfully driven financial inclusion initiatives across Indonesia.

Looking ahead, mobile phones and digital payment already play a big role in helping more people access financial services. Smartphones, mobile phones and digital payment apps are becoming the new bank tellers, ATMs, and loan officers all rolled into one. This shift is huge for both individuals and whole economies. This article will investigate how despite technological advancements, certain groups remain excluded from digital banking, and how we create the plan ahead to increase inclusivity.

\section*{The Challenges: Who's Left Behind?}
Imagine this, You're a single mom in rural Java, piecing together a living from odd jobs. Or a refugee in Aceh, clutching hope but no official documents. Congrats, you've just won a spot in the "financially invisible" club. It's not a club anyone wants to join, but millions are stuck there.

This isn't just about being far from the city or short on cash.

If you're scraping by on irregular work, living off the grid, dealing with a disability, or part of a community that's always been overlooked, good luck getting a bank account or a loan.

Overcoming these challenges and barriers to financial inclusion. isn't simple, Here's what we're also up against:

\begin{enumerate}
  \item Tech deserts: In rural places where electricity is scarce, talking about online banking sounds like science fiction. How can you go online without the internet?

  \item Digital newbies: It's like giving your grandmother a smartphone and expecting her to manage her savings on it. Millions have never used a computer, let alone online banking.

  \item Culture clash: In some communities, trusting an app with your money is very uncommon. Changing a cash-only mindset isn't just about introducing new technology—it's about building trust.

\end{enumerate}

\section*{The Road Ahead: What's the Initiatives?}
The good thing is, across indonesia, governments, and private sectors were are cooking up solutions:

\begin{enumerate}
  \item Homegrown companies like Gojek and Grab started with rides, but now they're financial powerhouses. Their apps (Gopay and OVO) offer loans, insurance, and more to people banks often ignore.

  \item Grassroots partnerships: Bank Rakyat Indonesia (BRI) from Badan Usaha Milik Negara (BUMN) is revolutionizing rural banking with its BRILink banking service. This innovative approach turns local shops and small businesses into mini-banks (they sign up as Agents), bringing financial services to the doorsteps of Indonesia's most remote communities.

\end{enumerate}

Meanwhile, these were 2 global comparisons that might influence and inspire on how indonesia take financial inclusivity more seriously on the government level and private sectors:

\begin{itemize}
  \item India's Unified Payments Interface (UPI) System: A government-led system enabling instant bank.transfers via mobile numbers, using it as simple as using text message (with SMS banking and USSD-based transactions). It processes millions monthly transactions, showcasing the potential of state-driven financial innovation.
\end{itemize}

The key difference is that UPI is a government-led initiative, while Indonesia's digital payment landscape is primarily driven by private companies.

\begin{itemize}
  \item WeChat Pay \& Alipay: China's super apps offer a glimpse of what Gopay and OVO might evolve into. Those two superapps have created entire financial ecosystems within social media platforms. Users can pay bills, invest in funds, and even get loans without leaving the app. While Indonesia's apps aren't quite at this level of integration yet, I believe we're moving in this direction.
\end{itemize}

\section*{Conclusion}
There is very clear evidence that promoting financial inclusion is a significant way to reduce poverty, build generational wealth, and increase economic prospects for billions of people, especially in indonesia. So my take on this: Collaboration is key. Public-to-private partnerships are bridging gaps where neither sector could succeed alone, the government should do more innovation too and need to step up as well as private sectors do these days. These collaborations are crucial in a country where geography often limits access to services.

So, what's coming next as trends that might shape Indonesia's financial landscape?:

\begin{enumerate}
  \item Encourage more of Regulatory Sandbox such as creating safe more spaces for fintech innovation, for encouraging more home-grown solutions to uniquely local problems

  \item AI and Machine Learning: These technologies promise to revolutionize automated credit scoring and risk assessment by incorporating a broader range of data sources like online behavior, potentially helping to assess creditworthiness based on mobile usage patterns instead of traditional credit scores.

  \item Edu-Finance: As financial services expand, financial education must keep pace. We need to step up more initiatives focused at boosting financial knowledge, especially in underserved communities.

\end{enumerate}

I believe Indonesia's got a chance to do something amazing here. If we get it right, millions of people could have better control over their money. We all know it won't be easy. There will be bumps in the road. But the potential payoff is huge: a fairer, more prosperous Indonesia where everyone has a shot at financial success, no matter where they live or how much they earn.

\section*{References}
China's Alipay and WeChat Pay: Reaching Rural Users. (2017, December 4). CGAP, from \href{https://www.cgap.org/sites/default/files/Brief-Chinas-Alipay-and-WeChat-Pay-Dec-2}{https://www.cgap.org/sites/default/files/Brief-Chinas-Alipay-and-WeChat-Pay-Dec-2} 017.pdf

Financial Inclusion Data. (n.d.). Financial Inclusion Data, from \href{https://datatopics.worldbank.org/financialinclusion/country/indonesia}{https://datatopics.worldbank.org/financialinclusion/country/indonesia}

Financial Inclusion - Overview, Barriers, Importance, \& the Rise of Fintech. (n.d.). Corporate Finance Institute, from \href{https://corporatefinanceinstitute.com/resources/economics/financial-inclusion/}{https://corporatefinanceinstitute.com/resources/economics/financial-inclusion/}

Finsights 2.0 - Indonesia. (2023, October 17). Dewan Nasional Keuangan Inklusif, from \href{https://snki.go.id/wp-content/uploads/2023/10/Youth-Finsights-2.0-Report.pdf}{https://snki.go.id/wp-content/uploads/2023/10/Youth-Finsights-2.0-Report.pdf}

OJK International Information Hub $\mid$ Otoritas Jasa Keuangan. (n.d.). OJK International Information Hub | Otoritas Jasa Keuangan, from\\
\href{https://www.ojk.go.id/iru/policy/detailpolicy/9625/press-release-2022-national-finan}{https://www.ojk.go.id/iru/policy/detailpolicy/9625/press-release-2022-national-finan} cial-literacy-and-inclusion-survey

OJK International Information Hub | Otoritas Jasa Keuangan. (2023, March 29). OJK

International Information Hub | Otoritas Jasa Keuangan, from

\href{https://www.ojk.go.id/iru/policy/detailpolicy/9965/press-release-ojk-issues-new-regu}{https://www.ojk.go.id/iru/policy/detailpolicy/9965/press-release-ojk-issues-new-regu}

lation-to-enhance-financial-inclusion-and-financial-literacy

Practical Guide to the WeChat Ecosystem in China. (2024, May 15). Woodburn Accountants \& Advisors from

\href{https://www.woodburnglobal.com/post/practical-guide-to-the-wechat-ecosystem-in-c}{https://www.woodburnglobal.com/post/practical-guide-to-the-wechat-ecosystem-in-c} hina

The Role of UPI in Financial Inclusion: Empowering the Unbanked. (n.d.). AU Small Finance Bank, from

\href{https://www.aubank.in/blogs/role-upi-financial-inclusion-empowering-unbanked}{https://www.aubank.in/blogs/role-upi-financial-inclusion-empowering-unbanked}


\end{document}
